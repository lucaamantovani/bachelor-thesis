\chapter{Introduzione}
\label{chap:Introduction}

Internet ha completamente rivoluzionato la quotidianità, lo stile di vita e il modo di farsi conoscere. Gran parte della popolazione trascorre la maggior parte del suo tempo online, alla ricerca di informazioni e contenuti appaganti che possano soddisfare determinate esigenze. Il livello d'interazione con i siti web risulta elevatissimo e l'esperienza maturata durante il tirocinio mi ha permesso di addentrarmi all'interno di questo ambiente e poterne apprendere al meglio il suo funzionamento e la sua importanza.
L'attività è stata svolta presso l'azienda mantovana Studio Indaco, specializzata in pianificazione strategica per il web, realizzazione siti internet, campagne di web marketing e fotografia. Durante la pratica ho avuto modo di poter interagire con Cubo CMS (il sistema per la gestione dei contenuti alla base della realizzazione dei siti da parte della società in questione) e studiare l'infrastruttura architettata per poter erogare e gestire al meglio i servizi messi a disposizione.
Il principale obiettivo del tirocinio è stato quello di approfondire la conoscenza dei linguaggi di programmazione applicati al web tramite l'implementazione del sistema per la gestione dei contenuti nell'ottica della produzione di diversi siti su specifiche richieste dei clienti. \hfill \break \break
Il documento mette in relazione tutti gli aspetti del CMS utilizzato in fase di progettazione e si struttura nel seguente modo: il capitolo 2 presenta un excursus per quanto ne riguarda la storia e l'evoluzione che ha avuto questo tipo di strumento.
Il capitolo 3 analizza la struttura ed il modo secondo il quale è stato sviluppato e implementato il sistema. Il capitolo 4 descrive le infrastrutture e le soluzioni messa in pratica per la distribuzione del servizio. Il capitolo 5 riassume l'operato e conclude il documento.